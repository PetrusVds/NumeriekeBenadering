\subsubsection{Metrische ruimte}

\vspace{0.5cm}

\begin{theo}[Afstand]{theo:afstand}
    Men zegt dat over een verzameling $A$ een afstand gedefinieerd is als er met elk paar elementen $x,y \in A$ een reëel getal $\rho(x,y)$ overeenstemt dat voldoet aan de volgende eigenschappen:
    \begin{itemize}
        \item Positief definitief: \ $\rho(x,y) \geq 0$ en $\rho(x,y) = 0 \ \Leftrightarrow \ x = y$
        \item Symmetrisch: \ $\rho(x,y) = \rho(y,x)$
        \item Driehoeksongelijkheid: \ $\rho(x,y) \leq \rho(z,x) + \rho(z,y)$
    \end{itemize}
    Een afstandsfunctie is bijgevolg een functionaal van de productverzameling $A \times A$ naar de verazmeling $\mathbb{R}$ van de reële getallen.
\end{theo}

\begin{theo}[Metrische ruimte]{theo:metrische_ruimte}
    De verzameling $A$ met een afstandsfunctie $\rho$ noemt men een \textbf{metrische ruimte} ($A,\rho$). De afstandsfunctie noemt men ook wel de \textbf{metriek} van de ruimte. De afstand tot een deelverzameling $D$ van een metrische ruimte wordt gedefinieerd als:
    \begin{equation*}
        \rho(x,D) = \inf\left\{\rho(x,y) \ | \ y \in D\right\}
    \end{equation*}
    \vspace{-0.5cm}
\end{theo}

\begin{theo}[Beste benadering]{theo:beste_benadering}
    Zij $D$ een deelverzameling van een metrische ruimte $(A,\rho)$. Een element $d \in D$ noemt men een beste benadering van een gegeven element $x\in A$, als er geen enkel ander element van $D$ dichter bij $x$ gelegen is dan $d$.
\end{theo}

\subsubsection{Vectorruimte}

\vspace{0.5cm}

\begin{theo}[Vectorruimte]{Vectorruimte}
    Een vectorruimte $V$ over het veld $\mathbb{F}$ is een verzameling van elementen waarop twee bewerkingen zijn gedefinieerd: optelling en een scalaire vermenigvuldiging met een element uit $\mathbb{F}$. Deze bewerkingen moeten voldoen aan volgende voorwaarden:
    \begin{enumerate}
        \item $\forall \vec{u},\vec{v} \in V: \ \vec{u} + \vec{v} \in V$
        \item $\forall \vec{u},\vec{v},\vec{w} \in V: \ \vec{u} + (\vec{v} + \vec{w}) = (\vec{u} + \vec{v}) + \vec{w}$
        \item Er bestaat een element $\vec{0} \in V$ zodat $\forall \vec{u} \in V: \ \vec{u} + \vec{0} = \vec{u}$
        \item Voor elke $\vec{v} \in V$ bestaat er element $-\vec{v} \in V$ zodat $\vec{v} + (-\vec{v}) = \vec{0}$
        \item $\forall \vec{u},\vec{v} \in V: \ \vec{u} + \vec{v} = \vec{v} + \vec{u}$
        \item $\forall \vec{v} \in V,\ \forall f \in \mathbb{F}: \ f \vec{v} \in V$
        \item $\forall \vec{v} \in V,\ \forall f,g \in \mathbb{F}: \ f(g\vec{v}) = (fg)\vec{v}$
        \item Als $1$ het eenheidselement is van $\mathbb{F}$, dan geldt $\forall \vec{v} \in V: \ 1\vec{v} = \vec{v}$
        \item $\forall \vec{u},\vec{v} \in V,\ \forall f \in \mathbb{F}: \ f(\vec{u} + \vec{v}) = f\vec{u} + f\vec{v}$
        \item $\forall \vec{v} \in V,\ \forall f,g \in \mathbb{F}: \ (f+g)\vec{v} = f\vec{v} + g\vec{v}$
    \end{enumerate}
\end{theo}

\begin{theo}[Norm en genormeerde ruimte]{theo:norm_ruimte}
    Een \textbf{norm} over de vectorruimte $V$ is een functionaal van $V$ naar $\mathbb{R}$ waarvan de beelden voldoen aan de volgende eigenschappen:
    \begin{itemize}
        \item Positief definiet: \ $\|\vec{x}\| \geq 0$ en $\|\vec{x}\| = 0 \ \Leftrightarrow \ \vec{x} = \vec{0}$
        \item Homogeniteit:\ $\|a\vec{x}\| = |a|\|\vec{x}\|$
        \item Driehoeksongelijkheid:\ $\|\vec{x} + \vec{y}\| \leq \|\vec{x}\| + \|\vec{y}\|$
    \end{itemize}
    Een vectorruimte waarover een norm gedefinieerd is, is een \textbf{genormeerde ruimte}.
\end{theo}

\begin{lem}[Metriciteit van de vectorruimte]{lem:metriciteit_vectorruimte}
    Als een vectorruimte genormeerd is, dan is ze ook metrisch. De functie
    \begin{equation*}
        \rho(\vec{x},\vec{y}) = \|\vec{x} - \vec{y}\|
    \end{equation*}
    voldoet aan de definitie van afstand.
\end{lem}

\newpage

\begin{prf}[Metriciteit van de vectorruimte]{prf:metriciteit_vectorruimte}
    De norm is per definitie positief definiet en triviaal symmetrisch. We willen nu aantonen dat het volgende geldt:
    \begin{equation*}
        \| \vec{x} - \vec{y} \| \leq \| \vec{x} - \vec{z} \| + \| \vec{y} - \vec{z} \|.
    \end{equation*}
    Welnu voor normen geldt per definitie de driehoeksongelijkheid:
    \begin{equation*}
        \|\vec{\alpha} + \vec{\beta}\| \leq \|\vec{\alpha}\| + \|\vec{\beta}\|;
    \end{equation*}
    stel hierin $\vec{\alpha} = \vec{x} - \vec{z}$ en $\vec{\beta} = \vec{z} - \vec{y}$, dan volgt:
    \begin{equation*}
        \|\vec{x} - \vec{y}\| = \|\vec{x} - \vec{z} + \vec{z} - \vec{y}\| \leq \|\vec{x} - \vec{z}\| + \|\vec{z} - \vec{y}\|,
    \end{equation*}
    waaruit de driehoeksongelijkheid volgt.
\end{prf}

\begin{lem}[Translatie-invariantie en homogeniteit]{lem:Translatie-invariantie en homogeniteit}
    Een metrische vectorruimte kan genormeerd worden met een norm die voldoet aan de definitie van afstand als en slechts als de afstandsfunctie voldoet aan:
    \begin{enumerate}
        \item Translatie-invariantie: $\forall \vec{x},\vec{y},\vec{z} \in V: \ \rho(\vec{x},\vec{y}) = \rho(\vec{x} + \vec{z},\vec{y} + \vec{z})$
        \item Homogeniteit: $\forall \vec{x},\vec{y} \in V,\ \forall a \in \mathbb{R}^+: \ \rho(a\vec{x},a\vec{y}) = a\rho(\vec{x},\vec{y})$
    \end{enumerate}
\end{lem}

\begin{theo}[Convexe en strikte convexe deelverzameling van een vectorruimte]{theo:convex_vectorruimte}
    Een deelverzameling $C$ van een vectorruimte $V$ is convex wanneer voor alle $\lambda > 0$ en $\mu >0$ met $\lambda + \mu = 1$ en voor alle $\vec{x},\vec{y} \in C$ geldt dat $\lambda \vec{x} + \mu \vec{y} \in C$. Wanneer al deze punten tit het inwendige van $C$ behoren, dan noemt men $C$ stikt convex.
\end{theo}

\begin{pro}[Convexiteit en genormeerde ruimte]{pro:convexiteit_genormeerde_ruimte}
    In een genormeerde ruimte is elke gesloten bol $B(\vec{a},r)$ convex.
\end{pro}

\begin{prf}[Convexiteit en genormeerde ruimte]{prf:convexiteit_genormeerde_ruimte}
    Inderdaad, zij $\vec{x}_1$ en $\vec{x}_2$ twee punten van $B(\vec{a},r)$. Dan moeten we aantonen dat $\lambda\vec{x}_1 + (1-\lambda)\vec{x}_2$ tot de bol behoort, of nog dat $\|\lambda\vec{x}_1 + (1-\lambda)\vec{x}_2 -\vec{a}\| \leq r$. Welnu:
    \begin{equation*}
        \|\lambda\vec{x}_1 + (1-\lambda)\vec{x}_2 -\vec{a}\| \leq \lambda \|\vec{x}_1 - \vec{a}\| + (1 - \lambda)\|\vec{x}_2 - \vec{a}\| \leq \lambda r + (1-\lambda)r = r.
    \end{equation*}
    \vspace{-0.5cm}
\end{prf}