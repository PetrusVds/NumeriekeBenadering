\subsubsection{Metrische ruimte}

\vspace{0.5cm}

\begin{theo}[Afstand]{theo:afstand}
    Men zegt dat over een verzameling $A$ een \emph{afstand} gedefinieerd is als er met elk paar elementen $x,y \in A$ een reëel getal $\rho(x,y)$ overeenstemt dat voldoet aan de volgende eigenschappen:
    \begin{itemize}
        \item Positief definitief: \ $\rho(x,y) \geq 0$ en $\rho(x,y) = 0 \ \Leftrightarrow \ x = y$
        \item Symmetrisch: \ $\rho(x,y) = \rho(y,x)$
        \item Driehoeksongelijkheid: \ $\rho(x,y) \leq \rho(z,x) + \rho(z,y)$
    \end{itemize}
    Een afstandsfunctie is bijgevolg een functionaal van de productverzameling $A \times A$ naar de verazmeling $\mathbb{R}$ van de reële getallen.
\end{theo}

\begin{theo}[Metrische ruimte]{theo:metrische_ruimte}
    De verzameling $A$ met een afstandsfunctie $\rho$ noemt men een \emph{metrische ruimte} ($A,\rho$). De afstandsfunctie noemt men ook wel de \emph{metriek} van de ruimte. De afstand tot een deelverzameling $D$ van een metrische ruimte wordt gedefinieerd als:
    \begin{equation*}
        \rho(x,D) = \inf\left\{\rho(x,y) \ | \ y \in D\right\}
    \end{equation*}
    \vspace{-0.5cm}
\end{theo}

\begin{theo}[Beste benadering]{theo:beste_benadering}
    Zij $D$ een deelverzameling van een metrische ruimte $(A,\rho)$. Een element $d \in D$ noemt men een beste benadering van een gegeven element $x\in A$, als er geen enkel ander element van $D$ dichter bij $x$ gelegen is dan $d$.
\end{theo}

\subsubsection{Vectorruimte}

\vspace{0.5cm}

\begin{theo}[Vectorruimte]{Vectorruimte}
    Een \emph{vectorruimte} $V$ over het veld $\mathbb{F}$ is een verzameling van elementen waarop twee bewerkingen zijn gedefinieerd: optelling en een scalaire vermenigvuldiging met een element uit $\mathbb{F}$. Deze bewerkingen moeten voldoen aan volgende voorwaarden:
    \begin{enumerate}
        \item $\forall \vec{u},\vec{v} \in V: \ \vec{u} + \vec{v} \in V$
        \item $\forall \vec{u},\vec{v},\vec{w} \in V: \ \vec{u} + (\vec{v} + \vec{w}) = (\vec{u} + \vec{v}) + \vec{w}$
        \item Er bestaat een element $\vec{0} \in V$ zodat $\forall \vec{u} \in V: \ \vec{u} + \vec{0} = \vec{u}$
        \item Voor elke $\vec{v} \in V$ bestaat er element $-\vec{v} \in V$ zodat $\vec{v} + (-\vec{v}) = \vec{0}$
        \item $\forall \vec{u},\vec{v} \in V: \ \vec{u} + \vec{v} = \vec{v} + \vec{u}$
        \item $\forall \vec{v} \in V,\ \forall f \in \mathbb{F}: \ f \vec{v} \in V$
        \item $\forall \vec{v} \in V,\ \forall f,g \in \mathbb{F}: \ f(g\vec{v}) = (fg)\vec{v}$
        \item Als $1$ het eenheidselement is van $\mathbb{F}$, dan geldt $\forall \vec{v} \in V: \ 1\vec{v} = \vec{v}$
        \item $\forall \vec{u},\vec{v} \in V,\ \forall f \in \mathbb{F}: \ f(\vec{u} + \vec{v}) = f\vec{u} + f\vec{v}$
        \item $\forall \vec{v} \in V,\ \forall f,g \in \mathbb{F}: \ (f+g)\vec{v} = f\vec{v} + g\vec{v}$
    \end{enumerate}
\end{theo}

\begin{theo}[Norm en genormeerde ruimte]{theo:norm_ruimte}
    Een \emph{norm} over de vectorruimte $V$ is een functionaal van $V$ naar $\mathbb{R}$ waarvan de beelden voldoen aan de volgende eigenschappen:
    \begin{itemize}
        \item Positief definiet: \ $\|\vec{x}\| \geq 0$ en $\|\vec{x}\| = 0 \ \Leftrightarrow \ \vec{x} = \vec{0}$
        \item Homogeniteit:\ $\|a\vec{x}\| = |a|\|\vec{x}\|$
        \item Driehoeksongelijkheid:\ $\|\vec{x} + \vec{y}\| \leq \|\vec{x}\| + \|\vec{y}\|$
    \end{itemize}
    Een vectorruimte waarover een norm gedefinieerd is, is een \emph{genormeerde ruimte}.
\end{theo}

\begin{lem}[Metriciteit van de vectorruimte]{lem:metriciteit_vectorruimte}
    Als een vectorruimte genormeerd is, dan is ze ook metrisch. De functie
    \begin{equation*}
        \rho(\vec{x},\vec{y}) = \|\vec{x} - \vec{y}\|
    \end{equation*}
    voldoet aan de definitie van afstand.
\end{lem}

\newpage

\begin{prf}[Metriciteit van de vectorruimte]{prf:metriciteit_vectorruimte}
    De norm is per definitie positief definiet en triviaal symmetrisch. We willen nu aantonen dat het volgende geldt:
    \begin{equation*}
        \| \vec{x} - \vec{y} \| \leq \| \vec{x} - \vec{z} \| + \| \vec{y} - \vec{z} \|.
    \end{equation*}
    Welnu voor normen geldt per definitie de driehoeksongelijkheid:
    \begin{equation*}
        \|\vec{\alpha} + \vec{\beta}\| \leq \|\vec{\alpha}\| + \|\vec{\beta}\|;
    \end{equation*}
    stel hierin $\vec{\alpha} = \vec{x} - \vec{z}$ en $\vec{\beta} = \vec{z} - \vec{y}$, dan volgt het gestelde, namelijk:
    \begin{equation*}
        \|\vec{x} - \vec{y}\| = \|\vec{x} - \vec{z} + \vec{z} - \vec{y}\| \leq \|\vec{x} - \vec{z}\| + \|\vec{z} - \vec{y}\|.
    \end{equation*}
    \vspace{-0.5cm}
\end{prf}

\begin{lem}[Translatie-invariantie en homogeniteit]{lem:Translatie-invariantie en homogeniteit}
    Een metrische vectorruimte kan genormeerd worden met een norm die voldoet aan de definitie van afstand als en slechts als de afstandsfunctie voldoet aan:
    \begin{enumerate}
        \item Translatie-invariantie: $\forall \vec{x},\vec{y},\vec{z} \in V: \ \rho(\vec{x},\vec{y}) = \rho(\vec{x} + \vec{z},\vec{y} + \vec{z})$
        \item Homogeniteit: $\forall \vec{x},\vec{y} \in V,\ \forall a \in \mathbb{R}^+: \ \rho(a\vec{x},a\vec{y}) = a\rho(\vec{x},\vec{y})$
    \end{enumerate}
\end{lem}

\begin{theo}[Convexe en strikte convexe deelverzameling van een vectorruimte]{theo:convex_vectorruimte}
    Een deelverzameling $C$ van een vectorruimte $V$ is convex wanneer voor alle $\lambda > 0$ en $\mu >0$ met $\lambda + \mu = 1$ en voor alle $\vec{x},\vec{y} \in C$ geldt dat $\lambda \vec{x} + \mu \vec{y} \in C$. Wanneer al deze punten tot het inwendige van $C$ behoren, dan noemt men $C$ stikt convex. Meetkundig betekent dit dat elk open lijnstuk $L(x,y)$ dat twee punten $x$ en $y$ van $C$ verbindt, volledig in $C$ ligt.
\end{theo}

\begin{pro}[Convexiteit en genormeerde ruimte]{pro:convexiteit_genormeerde_ruimte}
    In een genormeerde ruimte is elke gesloten bol $B(\vec{a},r)$ convex.
\end{pro}

\begin{prf}[Convexiteit en genormeerde ruimte]{prf:convexiteit_genormeerde_ruimte}
    Inderdaad, zij $\vec{x}_1$ en $\vec{x}_2$ twee punten van $B(\vec{a},r)$. Dan moeten we aantonen dat $\lambda\vec{x}_1 + (1-\lambda)\vec{x}_2$ tot de bol behoort, of nog dat $\|\lambda\vec{x}_1 + (1-\lambda)\vec{x}_2 -\vec{a}\| \leq r$. Welnu:
    \begin{equation*}
        \|\lambda\vec{x}_1 + (1-\lambda)\vec{x}_2 -\vec{a}\| \leq \lambda \|\vec{x}_1 - \vec{a}\| + (1 - \lambda)\|\vec{x}_2 - \vec{a}\| \leq \lambda r + (1-\lambda)r = r.
    \end{equation*}
    \vspace{-0.5cm}
\end{prf}

\begin{theo}[Strikt genormeerde ruimte en strike norm]{theo:strikte_norm_ruimte}
    Een genormeerde ruimte is strikt genormeerd als de eenheidsbol $B(\vec{0},1)$ strikt convex is. De eenheidsbol is strikt convex als er geen `rechte' lijnstukken in voorkomen, of wiskundig:
    \begin{equation*}
        \left( \vec{x} \neq \vec{y} \ \land \ \|\vec{x}\| = \| \vec{y} \| = 1 \right)
        \ \Rightarrow \ \|\vec{x} + \vec{y}\| < 2.
    \end{equation*}
    Men spreekt dan van een strikte norm. \\

    \textbf{Opmerking:} De 1-norm en de $\infty$-norm in $\mathbb{R}^n$ zijn \textbf{geen} strikte normen, omdat de eenheidsbol in deze normen niet strikt convex is.
\end{theo}

\begin{lem}[Beste benadering in een deelruimte]{lem:beste_benadering_deelruimte}
    Zij $\mathcal{D}$ een eindigdimensionale deelruimte van een strikt genormeerde ruimte $V$ en zij $\vec{v} \in V$. Dan bestaat de beste benadering van $\vec{v} \in \mathcal{D}$ en is deze uniek.  
\end{lem}

\begin{prf}[Beste benadering in een deelruimte]{prf:beste_benadering_deelruimte}
    \begin{itemize}
        \item 
            \textbf{Existentie}: Noem $d = \inf\{\|\vec{v}-\vec{w}\| \ | \ \vec{w} \in \mathcal{D}\}$. We tonen aan dat dit infimum in feite een minimum is. Volgens de definitie van een infimum bestaat er een rij van vectoren $\{\vec{w}_k\}_{k>1}$ in $\mathcal{D}$ zodat $\{\|\vec{v}-\vec{w}\|\}_{k>1}$ een dalende rij is die convergeert naar $d$. De rij $\{\vec{w}_k\}_{k>1}$ is bovendien uniform begrensd omdat 
            \begin{equation}
                \forall k > 1: \ \|\vec{w}_k\| = \|(\vec{w}_k - \vec{v}) + \vec{v}\| \leq \|\vec{w}_k - \vec{v}\| + \|\vec{v}\| \leq \|\vec{w}_1 - \vec{v}\| + \|\vec{v}\|
                \label{eq:uniform_begrensd}
            \end{equation}
            Stel $n$ gelijk aan de dimensie van $\mathcal{D}$ en beschouw een basis $\{\vec{a}_1,\ldots,\vec{a}_n\}$ van $\mathcal{D}$. Dan kunnen we $\vec{w}_k$ ontbinden als:
            \begin{equation*}
                \forall k > 1: \ \vec{w}_k = \sum_{i=1}^n \alpha_{ki}\vec{a}_i 
            \end{equation*}
            Uit (\ref{eq:uniform_begrensd}) volgt dat de rij $\{\alpha_{k1},\ldots,\alpha_{kn}\}_{k>1}$ uniform begrensd is. Deze rij heeft bijgevolg steeds een convergente deelrij (\emph{stelling van Weierstrass-Bolzano}) waarvan we de limiet $(\hat{\alpha}_1,\ldots,\hat{\alpha}_n)$ noemen. Daarom kunnen we, zonder algemeenheid in te boeten, in wat volgt veronderstellen dat de rij $\{\alpha_{k1},\ldots,\alpha_{kn}\}_{k>1}$ convergeert naar $(\hat{\alpha}_1,\ldots,\hat{\alpha}_n)$. Stel nu dat
            \begin{equation*}
                \vec{\zeta} = \sum_{i=1}^{n} \hat{\alpha}_i \vec{a}_i,
            \end{equation*} 
            dan geldt er voor alle $k\geq1$ dat 
            \begin{equation*}
                \|\vec{v} - \vec{\zeta}\| \leq \underbrace{\|\vec{v} - \vec{w}_k\|}_{\rightarrow d} + \underbrace{\|\vec{w}_k - \vec{\zeta}\|}_{\rightarrow 0},
            \end{equation*}
            wat $\|\vec{v} - \vec{\zeta}\| = d$ impliceert. De vector $\vec{\zeta}$ is bijgevolg de beste benadering van $\vec{v}$ in $\mathcal{D}$.
\newpage
        \item 
            \textbf{Uniciteit}: Het bewijs is uit het ongeruijmde. Veronderstel dat er twee verschillende beste benaderingen zijn, $\vec{\zeta}_1$ en $\vec{\zeta}_2$, zodat 
            \begin{equation*}
                \|\vec{v} - \vec{\zeta}_1\| = \|\vec{v} - \vec{\zeta}_2\| = d.
            \end{equation*}
            Merk dat $\vec{e}_i = \frac{1}{d}(\vec{v} - \vec{\zeta}_i)$ op de eenheidsbol in $V$ ligt voor $i = 1,2$. Omdat de eenheidsbol strikt convex is geldt
            \begin{equation*}
                \left\| \vec{v} - \frac{\vec{\zeta}_1 + \vec{\zeta}_2}{2} \right\| = d \left\| \underbrace{\frac{1}{2}}_\lambda\vec{e}_1 + \underbrace{\frac{1}{2}}_\mu\vec{e}_2 \right\| < d,
            \end{equation*}
            Dus $\frac{1}{2}(\vec{\zeta}_1 + \vec{\zeta}_2) \in \mathcal{D}$ is een betere benadering van $\vec{v}$ dan $\vec{\zeta}_1$, wat in tegenspraak is met de veronderstelling.
    \end{itemize}
\end{prf}

\subsubsection{Unitaire ruimte en orthogonaliteit}

\vspace{0.5cm}

\begin{theo}[Unitaire ruimte]{theo:unitaire_ruimte}
    Men noemt een vectorruimte $V$ over de complexe getallen unitair als er met elk paar elementen $\vec{x}, \vec{y} \in V$ een complex getal $(\vec{x},\vec{y})$ overeenstemt dat voldoet aan de volgende eigenschappen:
    \begin{enumerate}
        \item $\forall a \in \mathbb{C}: \ (\vec{x},a\vec{y}) = a(\vec{x},\vec{y})$
        \item $(\vec{x} + \vec{y},\vec{z}) = (\vec{x},\vec{z}) + (\vec{y},\vec{z})$
        \item $(\vec{x},\vec{y}) = \overline{(\vec{y},\vec{x})}$
        \item $(\vec{x},\vec{x}) > 0$ als $\vec{x} \neq \vec{0}$
    \end{enumerate}
    Men noemt $(\vec{x},\vec{y})$ het scalair product van $\vec{x}$ en $\vec{y}$. \\

    \textbf{Opmerking:} Uit de derde eigenschap volgt dat $(\vec{x},\vec{x})$ reëel is, sinds $(\vec{x},\vec{x}) = \overline{(\vec{x},\vec{x})}$. Hierdoor is het gebruik van het $'>'$-teken in de vierde eigenschap gerechtvaardigd. Ook, indien de vectorruimte reëel is, dan is de derde eigenschap symmetrisch.
\end{theo}

\begin{lem}[Unitair impliceert genormeerd]{lem:unitair_impliceert_genormeerd}
    Als een vectorruimte unitair is, dan is ze ook genormeerd. De functie
    \begin{equation*}
        \| \vec{x} \| = \sqrt{(\vec{x},\vec{x})}
    \end{equation*}
    voldoet aan de definitie van een norm. \\

    \textbf{Opmerking:} Deze norm wordt de \emph{natuurlijke} of \emph{geïnduceerde norm} genoemd. 
\end{lem}

\newpage

\begin{prf}[Unitair impliceert genormeerd]{prf:unitair_impliceert_genormeerd}
    De eerste `drie' normeigenschappen (positief definitief, homogeniteit) zijn gemakkelijk te bewijzen. De driehoeksongelijkheid volgt (voor $\vec{x} + \vec{y} \neq \vec{0}$) uit de Cauchy-Schwarz ongelijkheid:
    \begin{align*}
        (\vec{x} + \vec{y},\vec{x} + \vec{y}) 
            &= (\vec{x}, \vec{x} + \vec{y}) + (\vec{y}, \vec{x} + \vec{y}) \\
            &\leq \sqrt{(\vec{x},\vec{x})} \sqrt{(\vec{x} + \vec{y},\vec{x} + \vec{y})} + \sqrt{(\vec{y},\vec{y})} \sqrt{(\vec{x} + \vec{y},\vec{x} + \vec{y})} 
    \end{align*}
    en dus
    \begin{equation*}
        \sqrt{(\vec{x} + \vec{y},\vec{x} + \vec{y})} \leq \sqrt{(\vec{x},\vec{x})} + \sqrt{(\vec{y},\vec{y})}.
    \end{equation*}
    Voor het geval $\vec{x} + \vec{y} = \vec{0}$ is de driehoeksongelijkheid triviaal.
\end{prf}

\begin{lem}[Genormeerd naar unitair]{lem:genormeerd_naar_unitair}
    Een genormeerde vectorruimte is een unitaire ruimte met een scalair product dat voldoet aan Stelling~\ref{lem:unitair_impliceert_genormeerd}, als en slechts als de norm voldoet aan de \emph{parallellogramongelijkheid}:
    \begin{equation*}
        \| \vec{x} + \vec{y} \|^2 + \| \vec{x} - \vec{y} \|^2 = 2 \left(\| \vec{x} \|^2 + \| \vec{y} \|^2\right).
    \end{equation*}
    \textbf{Opnerking:} De zogenaamde parallellogramongelijkheid komt in het Euclidische vlak overeen met het feit dat de som van de kwadraten van de diagonalen van een parallellogram gelijk is aan de som van de kwadraten van de vier zijden.
\end{lem}

\begin{prf}[Genormeerd naar unitair]{prf:genormeerd_naar_unitair}
    Het nodig zijn wordt als volgt aangetoond:
    \begin{align*}
        \|\vec{x} + \vec{y}\|^2 + \|\vec{x} - \vec{y}\|^2 
            &= (\vec{x} + \vec{y},\vec{x} + \vec{y}) + (\vec{x} - \vec{y},\vec{x} - \vec{y}) \\
            &= 2(\vec{x},\vec{x}) + 2(\vec{y},\vec{y}).
    \end{align*}
    Voor een reële vecorruimte wordt het voldoende zijn bewezen door aan te tonen dat
    \begin{equation*}
        (\vec{x},\vec{y}) = \frac{1}{4}\left\{
            \| \vec{x} + \vec{y} \|^2 - \| \vec{x} \|^2 - \| \vec{y} \|^2
        \right\}
    \end{equation*}
    een scalair product is, en dat de natuurlijke norm van dit scalair product de oorspronkelijke norm is. Het bewijs is nogal technisch en laten we achterwege.
\end{prf}

\begin{lem}[Eenheidsbol in unitaire ruimte]{lem:eenheidsbol_unitaire_ruimte}
    De eenheidsbol in een unitaire ruimte is strikt convex.
\end{lem}

\newpage

\begin{prf}[Eenheidsbol in unitaire ruimte]{prf:eenheidsbol_unitaire_ruimte}
    Het volstaat aan te tonen dat de geziene formule in Definitie~\ref{theo:strikte_norm_ruimte} geldt. Welnu, neem $\vec{x} \neq \vec{y}$ met $\|\vec{x}\| = \|\vec{y}\| = 1$. Dan volgt uit de parallellogramongelijkheid
    \begin{equation*}
        \| \vec{x} + \vec{y} \|^2 = - \| \vec{x} - \vec{y} \|^2 + 2(\| \vec{x} \|^2 + \| \vec{y}\|^2)= - \| \vec{x} - \vec{y} \|^2 + 4 < 4.
    \end{equation*}
    En dus is $\| \vec{x} + \vec{y} \| < 2$.
\end{prf}

\begin{theo}[Orthogonaliteit]{theo:orthogonaliteit}
    Twee vectoren $\vec{x}$ en $\vec{y}$ in een unitaire ruimte $V$ zijn orthogonaal als hun scalair product nul is, of nog als:
    \begin{equation*}
        \vec{x} \perp \vec{y} \ \Rightarrow \ (\vec{x},\vec{y}) = 0.
    \end{equation*}
    \vspace{-0.5cm}
\end{theo}

\begin{lem}[Pythagoras]{lem:Pythagoras}
    Wanneer $\vec{x}$ en $\vec{y}$ orthogonaal zijn in een unitaire ruimte $V$, dan geldt t\@.o\@.v\@. de natuurlijke norm in $V$ dat
    \begin{equation*}
        \| \vec{x} + \vec{y} \|^2 = \| \vec{x} \|^2 + \| \vec{y} \|^2.
    \end{equation*}
    \vspace{-0.5cm}
\end{lem}

\begin{prf}[Pythagoras]{prf:Pythagoras}
    Het gestelde volgt triviaal uit een expansie van $\| \vec{x} + \vec{y} \|^2$, namelijk:
    \begin{align*}
        \| \vec{x} + \vec{y} \|^2 
            &= (\vec{x} + \vec{y},\vec{x} + \vec{y}) \\
            &= (\vec{x},\vec{x}) + (\vec{y},\vec{y}) + 2(\vec{x},\vec{y}) \\
            &= \| \vec{x} \|^2 + \| \vec{y} \|^2.
    \end{align*}
    \vspace{-1cm}
\end{prf}

\begin{lem}[Hermitiaans positief definiet]{lem:hermitiaans positief definiet}
    Neem de grammatrix $G$ van een stel vectoren $\vec{a}_1,\ldots,\vec{a}_n$, namelijk:
    \begin{equation*}
        G = G(\vec{a}_1,\ldots,\vec{a}_n) = 
        \begin{bmatrix}
            (\vec{a}_1,\vec{a}_1) & \cdots & (\vec{a}_1,\vec{a}_n) \\
            \vdots & \ddots & \vdots \\
            (\vec{a}_n,\vec{a}_1) & \cdots & (\vec{a}_n,\vec{a}_n)
        \end{bmatrix}.
    \end{equation*}
    Indien deze matrix Hermitiaans positief definiet is, dan zijn de vectoren $\vec{a}_1,\ldots,\vec{a}_n$ lineair onafhankelijk. Hieruit volgt dat de matrix $G$ een inverteerbare matrix is (als en slechts als de vectoren lineair onafhankelijk zijn).
\end{lem}

\begin{prf}[Hermitiaans positief definiet]{prf:hermitiaans positief definiet}
    \begin{itemize}
        \item 
            \textbf{Hermitiaans:} Uit de derde voorwaarde voor een scalair product (zoals gezien in Definitie~\ref{theo:unitaire_ruimte}) volgt dat $G = G^*$, dus G is een Hermitiaanse martix.
        \item 
            \textbf{Positief definitief:} Zij $v = \begin{bmatrix}
                v_1 & \cdots & v_n
            \end{bmatrix}$
            een vector in $\mathbb{C}^n$. Steunend op de lineairiteit van het scalair product kunnen we de uitdrukking $v^*Gv$ schrijven als:
            \begin{align*}
                v^*Gv 
                    &= \begin{bmatrix}
                        v_1 \\
                        \vdots \\
                        v_n
                    \end{bmatrix}^T
                    \begin{bmatrix}
                        (\vec{a}_1,\vec{a}_1) & \cdots & (\vec{a}_1,\vec{a}_n) \\
                        \vdots & \ddots & \vdots \\
                        (\vec{a}_n,\vec{a}_1) & \cdots & (\vec{a}_n,\vec{a}_n)
                    \end{bmatrix}
                    \begin{bmatrix}
                        v_1 \\ \vdots \\ v_n
                    \end{bmatrix} \\
                    &= \begin{bmatrix}
                        v_1 \\
                        \vdots \\
                        v_n
                    \end{bmatrix}^T
                    \begin{bmatrix}
                        (\vec{a}_1, v_1\vec{a}_1, + \cdots + v_n\vec{a}_n) \\
                        \vdots \\
                        (\vec{a}_n, v_1\vec{a}_1, + \cdots + v_n\vec{a}_n)
                    \end{bmatrix} \\
                    &= \left(\sum_{i=1}^n v_i\vec{a}_i, \sum_{j=1}^n v_j\vec{a}_j\right) 
                    = \left\| \sum_{i=1}^n v_i\vec{a}_i \right\|^2.
            \end{align*}
            Als $\{\vec{a}_1,\ldots,\vec{a}_n\}$ lineair onafhankelijk zijn, dan geldt voor $v\neq 0$ dat $\sum_{i=1}^n v_i\vec{a}_i \neq 0$ en dus $v^*Gv > 0$. Dit bewijst dat $G$ positief definiet is.
    \end{itemize}
\end{prf}

\begin{pro}[Orthogonale projector]{pro:orthogonale_projector}
    Een projector $P$ is orthogonaal als en alleen als:
    \begin{equation*}
        \forall \vec{v},\vec{w} \in V: \ (P\vec{v},\vec{w}) = (\vec{v},P\vec{w}).
    \end{equation*}
    \vspace{-0.5cm}
\end{pro}

\begin{prf}[Orthogonale projector]{prf:orthogonale_projector}
    ``$\Rightarrow$'': Als $\mathcal{R}(P)$ en $\mathcal{N}(P)$ orthogonaal zijn, dan is:
    \begin{equation*}
        \forall \vec{v},\vec{w} \in V: \ (P\vec{v},(I-P)\vec{w}) = \vec{0} = ((I-P)\vec{v},P\vec{w}),
    \end{equation*}
    waaruit volgt, wegens lineariteit van het scalair product, dat $(P\vec{v},\vec{w}) = (P\vec{v}, P\vec{w}) = (\vec{v},P\vec{w})$. \\
    
    ``$\Leftarrow$'': Neem willekeurige $\vec{x} = P\vec{u} \in \mathcal{R}(P)$ en $\vec{y} \in \mathcal{N}(P)$. Dan geldt:
    \begin{equation*}
        (\vec{x},\vec{y}) = (P\vec{u},\vec{y}) = (\vec{u},P\vec{y}) = (\vec{u},\vec{0}) = 0,
    \end{equation*}
    wat de orthogonaliteit bewijst.
\end{prf}

\newpage

\begin{alg}[Gram-Schmidt algoritem in een unitaire ruimte]{alg:gram_schmidt_unitaire_ruimte}
    \begin{tcolorbox}[colback=white, colframe=gray, arc=0mm] 
        \begin{algorithmic}[1]
        \For{$j = 1$ to $n$}
            \State $\vec{v}_j = \vec{a}_j$
            \For{$i = 1$ to $j - 1$}
                \State $r_{ij} = (\vec{q}_i,\vec{a}_j)$
                \State $\vec{v}_j = \vec{v}_j - r_{ij} \vec{q}_i$  \ ($= \vec{a}_j - P_{\vec{q}_1, \ldots, \vec{q}_{j-1}}\vec{a}_j$)
            \EndFor
            \State $r_{jj} = \|\vec{v}_j\|_2$
            \State $\vec{q}_j = \vec{v}_j / r_{jj}$
        \EndFor
        \end{algorithmic}
    \end{tcolorbox}
    \vspace{0.3cm}
    \textbf{Opmerking:} In vergelijking tot Algoritme~\ref{alg:Gram-Schmidt} zijn inwendige producten tussen vectoren in $\mathbb{C}^n$ vervangen door scalaire producten in $V$ en de Euclidische norm door de natuurlijke norm geïnduceerd door het scalair product, dit is $\|\cdot\| = \sqrt{(\cdot,\cdot)}$.
\end{alg}

\begin{lem}[Orthogonale projectiestelling]{lem:orthogonale_projectiestelling}
    De beste benadering van vector $\vec{v}$ in een deelruimte $\mathcal{D}$ van een unitaire ruimte $V$ met betrekking tot de natuurlijke norm wordt gegeven door $ \vec{\hat{y}} = P_{\mathcal{D}}\vec{v},$
    met $P_{\mathcal{D}}$ de orthogonale projector op $\mathcal{D}$. 
\end{lem}