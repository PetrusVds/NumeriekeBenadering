\subsubsection{Metrische ruimte}

\vspace{0.5cm}

\begin{theo}[Afstand]{theo:afstand}
    Men zegt dat over een verzameling $A$ een afstand gedefinieerd is als er met elk paar elementen $x,y \in A$ een reëel getal $\rho(x,y)$ overeenstemt dat voldoet aan de volgende eigenschappen:
    \begin{itemize}
        \item Positief definitief: \ $\rho(x,y) \geq 0$ en $\rho(x,y) = 0 \ \Leftrightarrow \ x = y$
        \item Symmetrisch: \ $\rho(x,y) = \rho(y,x)$
        \item Driehoeksongelijkheid: \ $\rho(x,y) \leq \rho(z,x) + \rho(z,y)$
    \end{itemize}
    Een afstandsfunctie is bijgevolg een functionaal van de productverzameling $A \times A$ naar de verazmeling $\mathbb{R}$ van de reële getallen.
\end{theo}

\begin{theo}[Metrische ruimte]{theo:metrische_ruimte}
    De verzameling $A$ met een afstandsfunctie $\rho$ noemt men een \textbf{metrische ruimte} ($A,\rho$). De afstandsfunctie noemt men ook wel de \textbf{metriek} van de ruimte. De afstand tot een deelverzameling $D$ van een metrische ruimte wordt gedefinieerd als:
    \begin{equation*}
        \rho(x,D) = \inf\left\{\rho(x,y) \ | \ y \in D\right\}
    \end{equation*}
    \vspace{-0.5cm}
\end{theo}

\begin{theo}[Beste benadering]{theo:beste_benadering}
    Zij $D$ een deelverzameling van een metrische ruimte $(A,\rho)$. Een element $d \in D$ noemt men een beste benadering van een gegeven element $x\in A$, als er geen enkel ander element van $D$ dichter bij $x$ gelegen is dan $d$.
\end{theo}

\subsubsection{Vectorruimte}

\vspace{0.5cm}

\begin{theo}[Vectorruimte]{Vectorruimte}
    Een vectorruimte $V$ over het veld $\mathbb{F}$ is een verzameling van elementen waarop twee bewerkingen zijn gedefinieerd: optelling en een scalaire vermenigvuldiging met een element uit $\mathbb{F}$. Deze bewerkingen moeten voldoen aan volgende voorwaarden:
    \begin{enumerate}
        \item $\forall \vec{u},\vec{v} \in V: \ \vec{u} + \vec{v} \in V$
        \item $\forall \vec{u},\vec{v},\vec{w} \in V: \ \vec{u} + (\vec{v} + \vec{w}) = (\vec{u} + \vec{v}) + \vec{w}$
        \item Er bestaat een element $\vec{0} \in V$ zodat $\forall \vec{u} \in V: \ \vec{u} + \vec{0} = \vec{u}$
        \item Voor elke $\vec{v} \in V$ bestaat er element $-\vec{v} \in V$ zodat $\vec{v} + (-\vec{v}) = \vec{0}$
        \item $\forall \vec{u},\vec{v} \in V: \ \vec{u} + \vec{v} = \vec{v} + \vec{u}$
        \item $\forall \vec{v} \in V,\ \forall f \in \mathbb{F}: \ f \vec{v} \in V$
        \item $\forall \vec{v} \in V,\ \forall f,g \in \mathbb{F}: \ f(g\vec{v}) = (fg)\vec{v}$
        \item Als $1$ het eenheidselement is van $\mathbb{F}$, dan geldt $\forall \vec{v} \in V: \ 1\vec{v} = \vec{v}$
        \item $\forall \vec{u},\vec{v} \in V,\ \forall f \in \mathbb{F}: \ f(\vec{u} + \vec{v}) = f\vec{u} + f\vec{v}$
        \item $\forall \vec{v} \in V,\ \forall f,g \in \mathbb{F}: \ (f+g)\vec{v} = f\vec{v} + g\vec{v}$
    \end{enumerate}
\end{theo}

\begin{theo}[Norm en genormeerde ruimte]{theo:norm_ruimte}
    Een \textbf{norm} over de vectorruimte $V$ is een functionaal van $V$ naar $\mathbb{R}$ waarvan de beelden voldoen aan de volgende eigenschappen:
    \begin{itemize}
        \item Positief definiet: \ $\|\vec{x}\| \geq 0$ en $\|\vec{x}\| = 0 \ \Leftrightarrow \ \vec{x} = \vec{0}$
        \item Homogeniteit:\ $\|a\vec{x}\| = |a|\|\vec{x}\|$
        \item Driehoeksongelijkheid:\ $\|\vec{x} + \vec{y}\| \leq \|\vec{x}\| + \|\vec{y}\|$
    \end{itemize}
    Een vectorruimte waarover een norm gedefinieerd is, is een \textbf{genormeerde ruimte}.
\end{theo}

\begin{lem}[Metriciteit van de vectorruimte]{lem:metriciteit_vectorruimte}
    Als een vectorruimte genormeerd is, dan is ze ook metrisch. De functie
    \begin{equation*}
        \rho(\vec{x},\vec{y}) = \|\vec{x} - \vec{y}\|
    \end{equation*}
    voldoet aan de definitie van afstand.
\end{lem}

\newpage

\begin{prf}[Metriciteit van de vectorruimte]{prf:metriciteit_vectorruimte}
    De norm is per definitie positief definiet en triviaal symmetrisch. We willen nu aantonen dat het volgende geldt:
    \begin{equation*}
        \| \vec{x} - \vec{y} \| \leq \| \vec{x} - \vec{z} \| + \| \vec{y} - \vec{z} \|.
    \end{equation*}
    Welnu voor normen geldt per definitie de driehoeksongelijkheid:
    \begin{equation*}
        \|\vec{\alpha} + \vec{\beta}\| \leq \|\vec{\alpha}\| + \|\vec{\beta}\|;
    \end{equation*}
    stel hierin $\vec{\alpha} = \vec{x} - \vec{z}$ en $\vec{\beta} = \vec{z} - \vec{y}$, dan volgt het gestelde, namelijk:
    \begin{equation*}
        \|\vec{x} - \vec{y}\| = \|\vec{x} - \vec{z} + \vec{z} - \vec{y}\| \leq \|\vec{x} - \vec{z}\| + \|\vec{z} - \vec{y}\|.
    \end{equation*}
    \vspace{-0.5cm}
\end{prf}

\begin{lem}[Translatie-invariantie en homogeniteit]{lem:Translatie-invariantie en homogeniteit}
    Een metrische vectorruimte kan genormeerd worden met een norm die voldoet aan de definitie van afstand als en slechts als de afstandsfunctie voldoet aan:
    \begin{enumerate}
        \item Translatie-invariantie: $\forall \vec{x},\vec{y},\vec{z} \in V: \ \rho(\vec{x},\vec{y}) = \rho(\vec{x} + \vec{z},\vec{y} + \vec{z})$
        \item Homogeniteit: $\forall \vec{x},\vec{y} \in V,\ \forall a \in \mathbb{R}^+: \ \rho(a\vec{x},a\vec{y}) = a\rho(\vec{x},\vec{y})$
    \end{enumerate}
\end{lem}

\begin{theo}[Convexe en strikte convexe deelverzameling van een vectorruimte]{theo:convex_vectorruimte}
    Een deelverzameling $C$ van een vectorruimte $V$ is convex wanneer voor alle $\lambda > 0$ en $\mu >0$ met $\lambda + \mu = 1$ en voor alle $\vec{x},\vec{y} \in C$ geldt dat $\lambda \vec{x} + \mu \vec{y} \in C$. Wanneer al deze punten tot het inwendige van $C$ behoren, dan noemt men $C$ stikt convex. Meetkundig betekent dit dat elk open lijnstuk $L(x,y)$ dat twee punten $x$ en $y$ van $C$ verbindt, volledig in $C$ ligt.
\end{theo}

\begin{pro}[Convexiteit en genormeerde ruimte]{pro:convexiteit_genormeerde_ruimte}
    In een genormeerde ruimte is elke gesloten bol $B(\vec{a},r)$ convex.
\end{pro}

\begin{prf}[Convexiteit en genormeerde ruimte]{prf:convexiteit_genormeerde_ruimte}
    Inderdaad, zij $\vec{x}_1$ en $\vec{x}_2$ twee punten van $B(\vec{a},r)$. Dan moeten we aantonen dat $\lambda\vec{x}_1 + (1-\lambda)\vec{x}_2$ tot de bol behoort, of nog dat $\|\lambda\vec{x}_1 + (1-\lambda)\vec{x}_2 -\vec{a}\| \leq r$. Welnu:
    \begin{equation*}
        \|\lambda\vec{x}_1 + (1-\lambda)\vec{x}_2 -\vec{a}\| \leq \lambda \|\vec{x}_1 - \vec{a}\| + (1 - \lambda)\|\vec{x}_2 - \vec{a}\| \leq \lambda r + (1-\lambda)r = r.
    \end{equation*}
    \vspace{-0.5cm}
\end{prf}

\begin{theo}[Strikt genormeerde ruimte en strike norm]{theo:strikte_norm_ruimte}
    Een genormeerde ruimte is strikt genormeerd als de eenheidsbol $B(\vec{0},1)$ strikt convex is. De eenheidsbol is strikt convex als er geen `rechte' lijnstukken in voorkomen, of wiskundig:
    \begin{equation*}
        \left( \vec{x} \neq \vec{y} \ \land \ \|\vec{x}\| = \| \vec{y} \| = 1 \right)
        \ \Rightarrow \ \|\vec{x} + \vec{y}\| < 2.
    \end{equation*}
    Men spreekt dan van een strikte norm. \\

    \textbf{Opmerking:} De 1-norm en de $\infty$-norm in $\mathbb{R}^n$ zijn \textbf{geen} strikte normen, omdat de eenheidsbol in deze normen niet strikt convex is.
\end{theo}

\begin{lem}[Beste benadering in een deelruimte]{lem:beste_benadering_deelruimte}
    Zij $\mathcal{D}$ een eindigdimensionale deelruimte van een strikt genormeerde ruimte $V$ en zij $\vec{v} \in V$. Dan bestaat de beste benadering van $\vec{v} \in \mathcal{D}$ en is deze uniek.  
\end{lem}

\begin{prf}[Beste benadering in een deelruimte]{prf:beste_benadering_deelruimte}
    \begin{itemize}
        \item 
            \textbf{Existentie}: Noem $d = \inf\{\|\vec{v}-\vec{w}\| | \vec{w} \in \mathcal{D}\}$. We tonen aan dat dit infimum in feite een minimum is. Volgens de definitie van een infimum bestaat er een rij van vectoren $\{\vec{w}_k\}_{k>1}$ in $\mathcal{D}$ zodat $\{\|\vec{v}-\vec{w}\|\}_{k>1}$ een dalende rij is die convergeert naar $d$. De rij $\{\vec{w}_k\}_{k>1}$ is bovendien uniform begrensd omdat 
            \begin{equation}
                \forall k > 1: \ \|\vec{w}_k\| = \|(\vec{w}_k - \vec{v}) + \vec{v}\| \leq \|\vec{w}_k - \vec{v}\| + \|\vec{v}\| \leq \|\vec{w}_1 - \vec{v}\| + \|\vec{v}\|
                \label{eq:uniform_begrensd}
            \end{equation}
            Stel $n$ gelijk aan de dimensie van $\mathcal{D}$ en beschouw $\{\vec{a}_1,\ldots,\vec{a}_n\}$ van $\mathcal{D}$. Dan kunnen we $\vec{w}_k$ met $k > 1$ ontbinden als:
            \begin{equation*}
                \vec{w}_k = \sum_{i=1}^n \alpha_{ki}\vec{a}_i 
            \end{equation*}
            Uit (\ref{eq:uniform_begrensd}) volgt dat de rij $\{\alpha_{k1},\ldots,\alpha_{kn}\}_{k>1}$ uniform begrensd is. Deze rij heeft bijgevolg steeds een convergente deelrij (\textbf{stelling van Weierstrass-Bolzano}) waarvan we de limiet $(\hat{\alpha}_1,\ldots,\hat{\alpha}_n)$ noemen. Daarom kunnen we, zonder algemeenheid in te boeten, in wat volgt veronderstellen dat de rij $\{\alpha_{k1},\ldots,\alpha_{kn}\}_{k>1}$ convergeert naar $(\hat{\alpha}_1,\ldots,\hat{\alpha}_n)$. Stel nu dat
            \begin{equation*}
                \vec{\zeta} = \sum_{i=1}^{n} \hat{\alpha}_i \vec{a}_i
            \end{equation*} 
            dan geldt er voor alle $k\geq1$ dat 
            \begin{equation*}
                \|\vec{v} - \vec{\zeta}\| \leq \underbrace{\|\vec{v} - \vec{w}_k\|}_{\rightarrow d} + \underbrace{\|\vec{w}_k - \vec{\zeta}\|}_{\rightarrow 0},
            \end{equation*}
            wat $\|\vec{v} - \vec{\zeta}\| = d$ impliceert. De vector $\vec{\zeta}$ is bijgevolg de beste benadering van $\vec{v}$ in $\mathcal{D}$.
\newpage
        \item 
            \textbf{Uniciteit}: Het bewijs is uit het ongeruijmde. Veronderstel dat er twee verschillende beste benaderingen zijn, $\vec{\zeta}_1$ en $\vec{\zeta}_2$, zodat 
            \begin{equation*}
                \|\vec{v} - \vec{\zeta}_1\| = \|\vec{v} - \vec{\zeta}_2\| = d.
            \end{equation*}
            Merk dat $\vec{e}_i = \frac{1}{d}(\vec{v} - \vec{\zeta}_i)$ op de eenheidsbol in $V$ ligt voor $i = 1,2$. Omdat de eenheidsbol strikt convex is geldt
            \begin{equation*}
                \left\| \vec{v} - \frac{\vec{\zeta}_1 + \vec{\zeta}_2}{2} \right\| = d \left\| \underbrace{\frac{1}{2}}_\lambda\vec{e}_1 + \underbrace{\frac{1}{2}}_\mu\vec{e}_2 \right\| < d,
            \end{equation*}
            Dus $\frac{1}{2}(\vec{\zeta}_1 + \vec{\zeta}_2) \in \mathcal{D}$ is een betere benadering van $\vec{v}$ dan $\vec{\zeta}_1$ en $\vec{\zeta}_2$, wat in tegenspraak is met de veronderstelling.
    \end{itemize}
\end{prf}

\subsubsection{Unitaire ruimte en orthogonaliteit}

\vspace{0.5cm}

\begin{theo}[Unitaire ruimte]
    Men noemt een vectorruimte $V$ over de complexe getallen unitair als er met elk paar elementen $\vec{x}, \vec{y} \in V$ een complex getal $(\vec{x},\vec{y})$ overeenstemt dat voldoet aan de volgende eigenschappen:
    \begin{enumerate}
        \item $\forall a \in \mathbb{C}: \ (\vec{x},a\vec{y}) = a(\vec{x},\vec{y})$
        \item $(\vec{x} + \vec{y},\vec{z}) = (\vec{x},\vec{z}) + (\vec{y},\vec{z})$
        \item $(\vec{x},\vec{y}) = \overline{(\vec{y},\vec{x})}$
        \item $(\vec{x},\vec{x}) > 0$ als $\vec{x} \neq \vec{0}$
    \end{enumerate}
    Men noemt $(\vec{x},\vec{y})$ het scalair product van $\vec{x}$ en $\vec{y}$. \\

    \textbf{Opmerking:} Uit de derde eigenschap volgt dat $(\vec{x},\vec{x})$ reëel is. Hierdoor is het scalair product over het veld $\mathbb{R}$ symmetrisch.
\end{theo}

\begin{lem}[Unitair impliceert genormeerd]{lem:unitair_impliceert_genormeerd}
    Als een vectorruimte unitair is, dan is ze ook genormeerd. De functie
    \begin{equation*}
        \| \vec{x} \| = \sqrt{(\vec{x},\vec{x})}
    \end{equation*}
    voldoet aan de definitie van een norm.
\end{lem}

\newpage

\begin{prf}[Unitair impliceert genormeerd]{prf:unitair_impliceert_genormeerd}
    De eerste `drie' normeigenschappen (positief definitief, homogeniteit) zijn gemakkelijk te bewijzen. De driehoeksongelijkheid volgt (voor $\vec{x} + \vec{y} \neq \vec{0}$) uit de Cauchy-Schwarz ongelijkheid:
    \begin{align*}
        (\vec{x} + \vec{y},\vec{x} + \vec{y}) 
            &= (\vec{x}, \vec{x} + \vec{y}) + (\vec{y}, \vec{x} + \vec{y}) \\
            &\leq \sqrt{(\vec{x},\vec{x})} \sqrt{(\vec{x} + \vec{y},\vec{x} + \vec{y})} + \sqrt{(\vec{y},\vec{y})} \sqrt{(\vec{x} + \vec{y},\vec{x} + \vec{y})} 
    \end{align*}
    en dus
    \begin{equation*}
        \sqrt{(\vec{x} + \vec{y},\vec{x} + \vec{y})} \leq \sqrt{(\vec{x},\vec{x})} + \sqrt{(\vec{y},\vec{y})}.
    \end{equation*}
    Voor het geval $\vec{x} + \vec{y} = \vec{0}$ is de driehoeksongelijkheid triviaal.
\end{prf}

\begin{lem}[Genormeerd naar unitair]{lem:genormeerd_naar_unitair}
    Een genormeerde vectorruimte is een unitaire ruimte met een scalair product dat voldoet aan Stelling~\ref{lem:unitair_impliceert_genormeerd}, als en slechts als de norm voldoet aan de parallellogramongelijkheid:
    \begin{equation*}
        \| \vec{x} + \vec{y} \|^2 + \| \vec{x} - \vec{y} \|^2 = 2 \left(\| \vec{x} \|^2 + \| \vec{y} \|^2\right).
    \end{equation*}
    \vspace{-0.5cm}
\end{lem}

\begin{prf}[Genormeerd naar unitair]{prf:genormeerd_naar_unitair}
    Het nodig zijn wordt als volgt aangetoond:
    \begin{align*}
        \|\vec{x} + \vec{y}\|^2 + \|\vec{x} - \vec{y}\|^2 
            &= (\vec{x} + \vec{y},\vec{x} + \vec{y}) + (\vec{x} - \vec{y},\vec{x} - \vec{y}) \\
            &= 2(\vec{x},\vec{x}) + 2(\vec{y},\vec{y}).
    \end{align*}
    Voor een reële vecorruimte wordt het voldoende zijn bewezen door aan te tonen dat
    \begin{equation*}
        (\vec{x},\vec{y}) = \frac{1}{4}\left\{
            \| \vec{x} + \vec{y} \|^2 - \| \vec{x} \|^2 - \| \vec{y} \|^2
        \right\}
    \end{equation*}
    een scalair product is, en dat de natuurlijke norm van dit scalair product de oorspronkelijke norm is. Het bewijs is nogal technisch en laten we achterwege.
\end{prf}

\begin{lem}[Eenheidsbol in unitaire ruimte]{lem:eenheidsbol_unitaire_ruimte}
    De eenheidsbol in een unitaire ruimte is strikt convex.
\end{lem}

\newpage

\begin{prf}[Eenheidsbol in unitaire ruimte]{prf:eenheidsbol_unitaire_ruimte}
    Het volstaat aan te tonen dat de geziene formule in Definitie~\ref{theo:strikte_norm_ruimte} geldt. Welnu, neem $\vec{x} \neq \vec{y}$ met $\|\vec{x}\| = \|\vec{y}\| = 1$. Dan volgt uit de parallellogramongelijkheid
    \begin{equation*}
        \| \vec{x} + \vec{y} \|^2 = - \| \vec{x} - \vec{y} \|^2 + 2(\| \vec{x} \|^2 + \| \vec{y}\|^2)= - \| \vec{x} - \vec{y} \|^2 + 4 < 4.
    \end{equation*}
    En dus is $\| \vec{x} + \vec{y} \| < 2$.
\end{prf}