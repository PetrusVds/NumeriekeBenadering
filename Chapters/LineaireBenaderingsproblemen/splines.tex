\subsubsection{Splines}

\vspace{0.5cm}

\begin{theo}[Splinefunctie]{theo:Splinefunctie}
    Zij een strikt stijgende rij van reële getallen gegeven, die voldoet aan
    \begin{equation*}
        a = t_o < t_1 < t_2 < \ldots < t_{n-1} < t_n = b.
    \end{equation*}
    Een spline functie $s(x)$ van graad $k > 0$ of orde $k+1$ met knooppunten $t_0,\ldots,t_n$ is een functie gedefinieerd op $[a,b]$ die voldoet aan de volgende eigenschappen:
    \begin{enumerate}
        \item in elk interval $[t_i,t_{i+1}]$ is $s(x)$ een veelterm van graad $k$ of lager;
        \item de functie $s(x)$ en haar afgeleiden tot en met orde $k-1$ zijn continu in $[a,b]$.
    \end{enumerate}
\end{theo}

\begin{lem}[Dimensie van de ruimte van splinefuncties]{lem:Dimensie_van_de_ruimte_van_splinefuncties}
    De vectorruimte van de splinefuncties van graad $k$ met knooppunten $t_0,\ldots,t_n$ heeft dimensie $n+k$.    
\end{lem}

\begin{app}[Interpolerende splinefunctie]{app:Interpolerende_splinefunctie}
    Een splinefunctie wordt \textbf{interpolerend} genoemd als hij in de knoopunten $t_0,\ldots,t_n$ bepaalde opgegeven waarden $f_0,\ldots,f_n$ aanneemt.
\end{app}

\begin{app}[Natuurlijke splinefunctie]{app:Natuurlijke_splinefunctie}
    Een \textbf{natuurlijke splinefunctie} is een splinefunctie van oneven graad $\exists m \geq 1:\ k = 2m+1$, waarvoor geldt dat 
    \begin{equation*}
        \forall j \in [m+1,2m]:\ s^{(j)}(a) = s^{(j)}(b) = 0.
    \end{equation*}
    \vspace{0.1cm}

    \textbf{Opmerking:} De meest courant gebruikte interpolerende splinefuncties zijn natuurlijke, interpolerende splinefuncties, d\@.w\@.z\@. dat dergelijke functie behoort tot oneven graad $k = 2m+1$ en voldoet aan:
    \begin{equation*}
        \forall j \in [0,m]:\ s^{(j)}(a) = s^{(j)}(b) = 0 \quad \text{en} \quad \forall i \in [0,n]:\ s(t_i) = f_i.
    \end{equation*}
    Dat zijn $(n+1) + 2m = n + k$ bijkomende voorwaarden en dus volgt, vermits de splimeruimte dimensie $n+k$ heeft (Stelling~\ref{lem:Dimensie_van_de_ruimte_van_splinefuncties}), dat deze functie eenduidig bepaald is.
\end{app}

\begin{app}[Periodieke splinefunctie]{app:Periodieke_splinefunctie}
    Een \textbf{periodieke splinefunctie} is een splinefunctie die voldoet aan
    \begin{equation*}
        \forall j \in [0,k-1]:\ s^{(j)}(a) = s^{(j)}(b).
    \end{equation*}
    \vspace{-0.5cm}
\end{app}

\subsubsection{B-splines}

\vspace{0.5cm}

\begin{theo}[Gedeelde differentie]{theo:Gedeelde_differentie}
    De gedeelde differentie van orde nul van een functie $f$ in een punt $x_i$ is gelijk aan 
    \begin{equation*}
        f[x_i] = f(x_i) = f_i.
    \end{equation*}
    De gedeelde differentie van orde $k=j-i$ voor $j>i$ van een functie $f$ in de verschillende punten $x_i,\ldots,x_j$ wordt gedefinieerd als 
    \begin{equation*}
        f[x_i,\ldots,x_j] = \frac{f[x_{i+1},\ldots,x_j] - f[x_i,\ldots,x_{j-1}]}{x_j - x_i} \left(= \Delta_t^k(x_i,\ldots,x_{i+k}) f(t)\right).
    \end{equation*}
    \vspace{-0.4cm}
\end{theo}

\begin{pro}[Gedeelde differentie]{pro:Gedeelde_differentie}
    \begin{enumerate}
        \item 
            De gedeelde differentie $f[x_i,\ldots,x_j]$ is lineair in $f$, d\@.w\@.z\@.
            \begin{equation*}
                (af + bg)[x_i,\ldots,x_j] = a f[x_i,\ldots,x_j] + b g[x_i,\ldots,x_j].
            \end{equation*}
        \item
            \textbf{Newton-vorm}: De interpolerende veelterm van graad $j-i$ door $(x_i,f_i), \ldots, (x_j,f_j)$ is gelijk aan:
            \begin{equation*}
                p_{j-i}(x) = a_0 + a_1(x-x_i) + a_2(x-x_i)(x-x_{i+1}) + \ldots + a_{j-i}\prod_{k=0}^{j-i-1}(x-x_{i+k}),
            \end{equation*}
            waarbij de coëfficiënten worden gegeven door:
            \begin{equation*}
                a_k = f[x_i,\ldots,x_{i+k}]
            \end{equation*}
        \item 
            De gedeelde differentie $f[x_i,\ldots,x_j]$ is continu in de argumenten $x_i,\ldots,x_j$ als $f(x)$ $(j-i)$-maal differentieerbaar is met continue $(j-i)$-de afgeleide.
        \item 
            De gedeelde differentie van orde $j-i$ van een veelterm $p_m(x)$ van graad $m$ met $m < j -i$, heeft de waarde nul, d\@.w\@.z\@. dat
            \begin{equation*}
                \exists m < j -i:\ p_m[x_i,\ldots,x_j] = 0.
            \end{equation*}
        \item 
            De gedeelde differentie $f[x_i,\ldots,x_j]$ is een lineaire samenstelling van de functie waarden $f_i,\ldots,f_j$, d\@.w\@.z\@. dat
            \begin{equation*}
                f[x_i,\ldots,x_j] = \sum_{k=i}^{j} \lambda_k f_k.
            \end{equation*}
        \item 
            De \textbf{formule van Leibniz} voor de gedeelde differentie van een product van twee functies luidt als volgt:
            \begin{equation*}
                f(x) = g(x)h(x) \ \Rightarrow \ f[x_i,\ldots,x_j] = \sum_{k=i}^{j} g[x_i,\ldots,x_k]h[x_k,\ldots,x_j].
            \end{equation*}
    \end{enumerate}
\end{pro}

\begin{theo}[Afgeknotte machtsfunctie]{theo:afgeknotte_machtsfunctie}
    Een veelgebruitke functie in de numerieke wiskunde is de afgeknotte machtsfunctie, die gedefinieerd is als
    \begin{equation*}
        (t-x)^k_+ = \begin{cases}
            (t-x)^k & \text{als } t > x, \\
            0 & \text{als } t \leq x.
        \end{cases}
    \end{equation*}    
\end{theo}

\begin{theo}[Gewone B-spline]{theo:Gewone_B-spline}
    De gewone B-spline van graad $k$ of orde $k+1$ wordt gegeven door:
    \begin{equation*}
        M_{i,k+1}(x) = \Delta_t^{k+1}(t_i,\ldots,t_{i+k+1}) (t-x)^k_+.
    \end{equation*}
    \vspace{-0.3cm}
\end{theo}

\begin{theo}[Genormaliseerde B-spline]{theo:Genormaliseerde_B-spline}
    De Genormaliseerde B-spline van graad $k$ of orde $k+1$ wordt gegeven door:
    \begin{equation*}
        N_{i,k+1}(x) = (t_{i+k+1} - t_i) M_{i,k+1}(x).
    \end{equation*}
    \vspace{-0.3cm}
\end{theo}

\begin{lem}[Geldigheid van B-splines]{lem:Geldigheid_van_B-splines}
    De gewone B-splinefunctie $M_{i,k+1}(x)$ en de genormaliseerde B-splinefunctie $N_{i,k+1}(x)$ zijn splinefuncties.
\end{lem}

\begin{prf}[Geldigheid van B-splines]{prf:Geldigheid_van_B-splines}
    Herneem de eigenschap dat de gedeelde differentie een lineaire samenstelling is van functiewaarden $f_i,\ldots,f_j$ (zoals gezien in Propositie~\ref{pro:Gedeelde_differentie}), namelijk:
    \begin{equation*}
        f[x_i,\ldots,x_j] = \sum_{k=i}^{j} \lambda_k f_k.
    \end{equation*}
    De functie $M_{i,k+1}(x)$ is dus een lineaire samenstelling van afgeknotte-machtsfuncties van graad $k$ van de volgende vorm:
    \begin{equation*}
        1, x, x^2, \ldots, x^k, (t_1 - x)^k_+, \ldots, (t_{n-1} - x)^k_+.
    \end{equation*}
    Deze lineaire samenstelling vormt een basis van de vectorruimte van de splinefuncties van graad $k$ met knooppunten $t_0,\ldots,t_n$ en dus is $M_{i,k+1}(x)$ een splinefunctie. Hetzelfde geldt voor $N_{i,k+1}(x)$.
\end{prf}

\begin{pro}[B-spline - Eerste orde]{pro:B-splines-1}
    \begin{equation*}
        M_{i,1}(x) = \begin{cases}
            \frac{1}{t_{i+1}-t_i} & \text{als } t_i \leq x < t_{i+1}, \\
            0 & \text{anders},
        \end{cases}, \quad 
        N_{i,1}(x) = \begin{cases}
            1 & \text{als } t_i \leq x < t_{i+1}, \\
            0 & \text{anders}.
        \end{cases}
    \end{equation*}
    % Voor een gegeven $i$ zijn de B-splines van orde 1 als volgt gedefinieerd:
    % \begin{itemize}
    %     \item $M_{i,1}(x) = \begin{cases}
    %         \frac{1}{t_{i+1}-t_i} & \text{als } t_i \leq x < t_{i+1}, \\
    %         0 & \text{anders},
    %     \end{cases}$
    %     \item $N_{i,1}(x) = \begin{cases}
    %         1 & \text{als } t_i \leq x < t_{i+1}, \\
    %         0 & \text{anders}.
    %     \end{cases}$
    % \end{itemize}
\end{pro}

\begin{prf}[B-spline - Eerste orde]{prf:B-splines-1}
    Per definitie van $M_{i,k+1}(x)$ geldt dat
    \begin{align*}
        M_{i,1}(x) 
            &= \Delta_t^1(t_i,t_{i+1}) (t-x)^0_+ \\
            &= \frac{1}{t_{i+1}-t_i} (t-x)_+ \\
            &= \frac{(t_{i+1} - x)^0_+ - (t_i - x)^0_+}{t_{i+1}-t_i}.
    \end{align*} 
    Het bovenste deel van de breuk is gelijk aan 1 als $t_i \leq x < t_{i+1}$ en 0 anders, hierdoor volgt dus:
    \begin{equation*}
        M_{i,1}(x) = \begin{cases}
            \frac{1}{t_{i+1}-t_i} & \text{als } t_i \leq x < t_{i+1}, \\
            0 & \text{anders}.
        \end{cases}
    \end{equation*}
    Hieruit volgt ook $N_{i,1}(x)$, sinds dit neer komt op vermenigvuldigen met $(t_{i+1} - t_i)$. \\

    \textbf{Opmerking:}  Dit bewijs dient niet gekend te zijn, enkel grafisch inzien.
\end{prf}

\begin{pro}[B-spline]{pro:B-splines-2}
    % Voor een gegeven $i$ en $k\geq 1$, geldt voor $x \leq t_i$ of $x \geq t_{i+k+1}$ dat
    \begin{equation*}
    k\geq 1,\ x \in \Omega:\ M_{i,k+1}(x) = 0 \quad \text{met} \quad \Omega = (-\infty, t_i] \cap [t_{i+k+1}, \infty)
    \end{equation*}
    \vspace{-.2cm}
\end{pro}

\begin{prf}[B-spline]{prf:B-splines-2}
    We leidden vroeger reeds af dat 
    \begin{equation*}
        M_{i,k+1}(x) = \sum_{s=i}^{i+k+1} \lambda_s f_s \quad \text{met} \quad f_s = (t_s - x)^k_+.
    \end{equation*}
    Wanneer nu $x \geq t_{i+k+1}$, dan zijn alle $f_s$ in bovenstaande formule nul. Als $x \leq t_i$, dan mogen we in de uitdrukking voor $f_s$ het vervangen door equivalente $(t-x)^k$ en dit voor elke $s$. De gewone B-spline wordt dan:
    \begin{equation*}
        M_{i,k+1}(x) = \Delta_t^{k+1} (t_i,\ldots,t_{i+k+1}) (t-x)^k.
    \end{equation*}
    Dat is identiek nul, want differentie van orde $k+1$ avn een veelterm van graad $k$ is nul. \\

    \textbf{Opmerking:}  Dit bewijs dient niet gekend te zijn, enkel grafisch inzien.
\end{prf}

\begin{pro}[B-spline - Recursiebetrekkingen]{pro:B-splines-3}
    \begin{itemize}
        \item $M_{i,k+1}(x) = \frac{x-t_i}{t_{i+k+1}-t_i} M_{i,k}(x) + \frac{t_{i+k+1}-x}{t_{i+k+1}-t_{i}} M_{i+1,k}(x)$,
        \item $N_{i,k+1}(x) = \frac{x-t_i}{t_{i+k}-t_i} N_{i,k}(x) + \frac{t_{i+k+1}-x}{t_{i+k+1}-t_{i+1}} N_{i+1,k}(x)$.
    \end{itemize}
\end{pro}

\begin{prf}[B-spline - Recursiebetrekkingen]{prf:B-splines-3}
    Voor $k \geq 1$ kunnen we schrijven dat:
    \begin{equation*}
        (t-x)^k_+ = (t-x)^{k-1}_+ (t-x).
    \end{equation*}
    We vullen dit in in de definitie van $M_{i,k+1}(x)$ en passen de formule van Leibniz voor gedeelde differentie  (zie Eigenschap~\ref{pro:Gedeelde_differentie}) toe, namelijk:
    \begin{equation*}
        f(x) = g(x)h(x) \ \Rightarrow \ f[x_i,\ldots,x_j] = \sum_{k=i}^{j} g[x_i,\ldots,x_k]h[x_k,\ldots,x_j].
    \end{equation*}
    Omwille van de factor $(t-x)$, een veelterm van graad 1 in $t$, bevatten de meeste termen in de bovenstaande som een factor die gelijk is aan nul. We vinden:
    \begin{align*}
        M_{i_k+1}(x) 
            &= \Delta_t^{k+1}(t_i,\ldots,t_{i+k+1}) (t-x)^k_+ \\
            &= \Delta_t^{k+1}(t_i,\ldots,t_{i+k+1}) \{(t-x)^{k-1}_+ (t-x)\} \\
            &= \Delta_t^{k+1}(t_i,\ldots,t_{i+k+1}) (t-x)^{k-1}_+ \cdot \Delta^0_t(t_{i+k+1})(t-x) + \\ 
            &   \quad \quad \Delta_t^{k}(t_i,\ldots,t_{i+k})(t-x)^{k-1}_+ \cdot \Delta^1_t(t_{i+k},t_{i+k+1})(t-x) \\
            &= \Delta_t^{k+1}(t_i,\ldots,t_{i+k+1}) (t-x)^{k-1}_+(t_{i+k+1}-x) + M_{i,k}(x) 
    \end{align*}
    Men gebruikt nu de recursieve definitie van gedeelde differentie, om de differentie van orde $k+1$ in bovenstaande uitdrukking te schrijven als een lineaire combinatie van twee gedeelde differenties van orde $k$. Rekening houdend met de definitie van B-spline, verkrijgt men:
    \begin{equation*}
        M_{i,k+1}(x) =  \frac{M_{i+1,k}(x) - M_{i,k}(x)}{t_{i+k+1}-t_i}(t_{i+k+1}-x) + M_{i,k}(x).
    \end{equation*}
    Hieruit volgt de recursiebetrekking voor $M_{i,k+1}(x)$. De recursiebetrekking voor $N_{i,k+1}(x)$ volgt op analoge wijze. \\

    \textbf{Opmerking:}  Dit bewijs dient niet gekend te zijn, enkel grafisch inzien.
\end{prf}

\begin{pro}[B-spline]{pro:B-splines-4}
    \begin{equation*}
        k\geq 1,\ x \in (t_i,t_{i+k+1}): \ M_{i,k+1}(x) > 0
    \end{equation*}
    \vspace{-0.2cm}
\end{pro}

\begin{prf}[B-spline]{prf:B-splines-4}
    We geven een bewijs gebaseerd op de recursiebetrekking en maken gebruik van volledige inductie. \\ 
            
    Voor $k=1$ luidt de recursiebetrekking
    \begin{equation*}
        M_{i,2}(x) = \frac{x-t_i}{t_{i+2}-t_i} M_{i,1}(x) + \frac{t_{i+2}-x}{t_{i+2}-t_{i+1}} M_{i+1,1}(x).
    \end{equation*}
    Schrijven we het rechterlid als $AB + CD$, dan is $A >0$ voor $x \in (t_i,\infty)$, $B>0$ voor $x\in [t_i,t_{i+1})$ en nul daarbuiten, $C>0$ voor $x \in (-\infty,t_{i_2})$, $D>0$ voor $x\in [t_{i_1},t_{i+2})$ en nul daarbuiten. Uit dit alles volgt dat $M_{i,2}(x) >0$ als $x \in (t_i,t_{i+2})$.  \\

    Voor de inductie stap bekijken we de recursiebetrekking $M_i^{k+1}$,
    \begin{equation*}
        M_{i,k+1}(x) = \frac{x-t_i}{t_{i+k+1}-t_i} M_{i,k}(x) + \frac{t_{i+k+1}-x}{t_{i+k+1}-t_{i}} M_{i+1,k}(x).
    \end{equation*}
    Schrijven we ook hier het rechterlid als $AB + CD$, dan is $A >0$ voor $x \in (t_i,\infty)$, $B>0$ als $x\in [t_i,t_{i+k})$ en nul daarbuiten, $C>0$ voor $x \in (-\infty,t_{i+k+1})$, $D>0$ als $x\in [t_{i+k},t_{i+k+1})$ en nul daarbuiten.  Er volgt dat $M_{i,k+1}(x) >0$ als $x \in (t_i,t_{i+k+1})$. \\

    \textbf{Opmerking:}  Dit bewijs dient niet gekend te zijn, enkel grafisch inzien.
\end{prf}

\begin{pro}[B-spline]{pro:B-splines-5}
    \begin{equation*}
        k \geq 1,\ \forall j \in [0,k-1]:\ M^{(j)}_{i,k+1}(t_i) =  M^{(j)}_{i,k+1}(t_{i+k+1}) = 0 
    \end{equation*}
    \vspace{-0.2cm}
\end{pro}

\begin{prf}[B-spline]{prf:B-splines-5}
    Sinds we bewezen hebben dat een B-spline een geldige splinefunctie is (zie Bewijs~\ref{prf:Geldigheid_van_B-splines}), volgt het te bewijzen uit de tweede eigenschap van splinefuncties in tandem met Eigenschap~\ref{pro:B-splines-2}. \\

    \textbf{Opmerking:}  Dit bewijs dient niet gekend te zijn, enkel grafisch inzien.
\end{prf}

\begin{pro}[B-spline]{pro:B-splines-6}
    \begin{equation*}
        k \geq 1,\ x \in [t_0,t_n]:\ \sum_{i=-k}^{n-1} N_{i, k+1}(x) = 1
    \end{equation*}
    \vspace{-0.2cm}
\end{pro}

\begin{prf}[B-spline]{prf:B-splines-6}
    We gebruiken de recursiebetrekking en Eigenschap~\ref{prf:B-splines-2} om aan te tonen dat $x \in [t_j,t_{j+1})$ geldt:
    \begin{align*}
        \sum_{i=-k}^{n-1} N_{i, k+1}(x) 
            &= \sum_{i=j-k}^{j} N_{i, k+1}(x) \\
            &= \sum_{i=j-k}^{j} \left\{
                \frac{x-t_i}{t_{i+k} - t_i} N_{i,k}(x) + \frac{t_{i+k+1}-x}{t_{i+k+1}-t_{i+1}} N_{i+1,k}(x)
            \right\} \\
            &= \frac{x - t_{j-k}}{t_j - t_{j-k}} N_{j-k,k}(x) 
            % \\    &\quad \quad \quad \quad \quad 
                + \sum_{i=j-(k-1)}^j \left\{
                \frac{x-t_i}{t_{i+k} - t_i} + \frac{t_{i+k}-x}{t_{i+k}-t_{i}}
                    \right\} N_{i,k}(x) \\
                &\quad \quad \quad \quad \quad + \frac{t_{j+k+1}-x}{t_{j+k+1}-t_{j+1}} N_{j+1,k}(x) 
    \end{align*}
    Uit Eigenschap~\ref{prf:B-splines-2} volgt dat $N_{j-k,k}(x)$ en $N_{j+1,k}(x)$ identiek nul zijn voor $x \in [t_j,t_{j+1}]$. Dus:
    \begin{equation*}
        \sum_{i=j-k}^{j} N_{i,k+1}(x) = \sum_{i=j-(k-1)}^j N_{i,k}(x).
    \end{equation*} 
    We kunnen op analoge manier verder gaan. We vinden 
    \begin{equation*}
        \sum_{i=j-k}^j N_{i,k+1}(x) = \ldots = \sum_{i=j-1}^{j} N_{i,2}(x) = N_{j,1}(x).
    \end{equation*}
    Voor $x\in[t_j,t_{j+1})$ is het rechterlid gelijk aan 1. Dat bewijst de stelling voor alle $x\in[t_0,t_n)$. Het geval $x = t_n$ volgt uit de continuïteit van de functie
    \begin{equation*}
        k \geq 1:\ \sum_{i=-k}^{n-1} N_{i, k+1}(x).
    \end{equation*}

    \textbf{Opmerking:}  Dit bewijs dient niet gekend te zijn, enkel grafisch inzien.
\end{prf}

\begin{pro}[B-spline]{pro:B-splines-7}
    % De (rechter)afgeleide van $N_{i,k+1}(x)$ wordt gegeven door:
    \begin{equation*}
        k \geq 1:\ N'_{i,k+1}(x) = k\left(
            \frac{N_{i,k}(x)}{t_{i+k}-t_i} 
                - \frac{N_{i+1,k}(x)}{t_{i+k+1}-t_{i+1}}
        \right)
    \end{equation*}
    \vspace{-0.2cm}
\end{pro}

\begin{prf}[B-spline]{pro:B-splines-7}
    In het aanstaande bewijs zal gebruik gemaakt worden van volgende ongeziene eigenschap van de afgeknotte machtsfunctie:
    % \begin{pro}[Afgeknotte machtsfunctie]{pro:Afgeknotte_machtsfunctie}
        \begin{equation*}
            \frac{d}{dx} (t-x)^k_+ = -k(t-x)^{k-1}_+.
        \end{equation*}
    %     \vspace{-0.2cm}
    % \end{pro}
    Deze formule is steeds geldig, uitgezonderd voor het berekenen van de afgeleide in het punt $x=t$ voor $k=1$. In dat geval geeft de formule de waarde van de rechterafgeleide. \\

    We bewijzen nu het gestelde:
    \begin{align*}
        N'_{i,k+1}(x) 
            &= \frac{d}{dx}\left\{
                    (t_{i+k+1}-t_i) \Delta_t^{k+1}(t_i,\ldots,t_{i+k+1}) (t-x)^k_+
                \right\} \\
            &= (t_{i+k+1}-t_i) \Delta_t^{k+1}(t_i,\ldots,t_{i+k+1}) \frac{d}{dx} (t-x)^k_+ \\
            &= -k(t_{i+k+1}-t_i) \Delta_t^{k+1}(t_i,\ldots,t_{i+k+1}) (t-x)^{k-1}_+ \\
            &= -k\left(
                \Delta_t^k(t_{i+1},\ldots,t_{i+k+1})(t-x)_+^{k-1}
                - \Delta_t^k(t_i,\ldots,t_{i+k})(t-x)_+^{k-1}
            \right) \\
            &= k\left(
                \frac{N_{i,k}(x)}{t_{i+k}-t_i} 
                - \frac{N_{i+1,k}(x)}{t_{i+k+1}-t_{i+1}}
            \right).
    \end{align*}
    \textbf{Opmerking:}  Dit bewijs dient niet gekend te zijn, enkel grafisch inzien.
\end{prf}

\begin{pro}[Splinefunctie - B-splinevoorstelling]{pro:B-splinevoorstelling}
    De splinefunctie $s(x)$ kan worden voorgesteld als een lineaire combinatie van $n+k$ B-splines,
    \begin{equation*}
        s(x) = \sum_{i=-k}^{n-1} c_i N_{i,k+1}(x).
    \end{equation*}
    \vspace{-0.3cm}
\end{pro}

\begin{lem}[de Boor]{lem:de_Boor}
    Zij $x\in[t_j,t_{j+1})$. Dan geldt $s(x) = c_j^{[k]}$. De constante $c_i^{[0]} = c_i$ en $c_i^{[r]}$ wordt gevonden uit 
    \begin{equation*}
        c_i^{[r]} = \alpha_{i,r} c_i^{[r-1]} + (1-\alpha_{i,r})c_{i-1}^{[r-1]} \quad \text{met} \quad \alpha_{i,r} = \frac{x-t_i}{t_{i+k+1-r}-t_i}.
    \end{equation*}
    \vspace{-0.2cm}
\end{lem}

\begin{prf}[de Boor]{prf:de_Boor}
    We schrijven $s(x)$ als een lineaire combinatie van B-splines van orde $k+1$, waarna we Eigenschap~\ref{pro:B-splines-3} hanteren en vereenvoudigen:
    \begin{align*}
        s(x) 
            &= \sum_{i=-k}^{n-1}c_i N_{i,k+1}(x) \\
            &= \sum_{i=-k}^{n-1}c_i
                \left(
                    \frac{x-t_i}{t_{i+k}-t_i} N_{i,k}(x) 
                    + \frac{t_{i+k+1} - x}{t_{i+k+1} - t_{i+1}} N_{i+1,k}(x)
                \right) \\
            &= \sum_{i=-(k-1)}^{n-1} c_i \frac{x-t_i}{t_{i+k} - t_i} N_{i,k}(x) 
                + \sum_{i=-k}^{n-2} c_i \frac{t_{i+k+1} - x}{t_{i+k+1} - t_{i+1}} N_{i+1,k}(x) \\
            &= \sum_{i=-(k-1)}^{n-1} 
                \left(
                    c_i \frac{x-t_i}{t_{i+k} - t_i} + c_{o-1} \frac{t_{i+k} - x}{t_{i+k} - t_i}
                \right)
                N_{i,k}(x) \\
            &= \sum_{i=-(k-1)}^{n-1} c_i^{[1]} N_{i,k}(x).
    \end{align*}
    Schrijven we $c_i^{[0]} = c_i$, dan vinden we de $c_i^{[1]}$-coëfficiënten als
    \begin{equation*}
        c_i^{[1]} = \alpha_{i,1} c_i^{[0]} + (1-\alpha_{i,1})c_{i-1}^{[0]} \quad \text{met} \quad \alpha_{i,1} = \frac{x-t_i}{t_{i+k}-t_i}. 
    \end{equation*} 
    Op een analoge manier kunnen we verder gaan en vinden dat 
    \begin{equation*}
        s(x) = \sum_{i=-(k-r)}^{n-1} c_i^{[r]}N_{i,k+1 - r}(x) = \ldots = \sum_{i=o}^{n-1} c_i^{[k]}N_{i,1}(x).
    \end{equation*}
    waarbij $c_i^{[r]}$ gegeven wordt door de recursiebetrekking van het gestelde. Voor $x\in[t_j,t_{j+1})$ zijn alle B-splines van eerste orde gelijk aan nul, op $N_{j,1}(x)$ na. Die neemt er de waarde $1$ aan. We vinden dus:
    \begin{equation*}
        s(x) = c_j^{[k]}.
    \end{equation*}
    \vspace{-0.8cm}
\end{prf}

\begin{lem}[Differentiëren van een splinefunctie]{lem:diff_spline}
    De afgeleide van orde $r$ van een splinefunctie $s(x)$ voldoet aan
    \begin{equation*}
        s^{(r)}(x) = \sum_{i=-(k-r)}^{n-1}c_i^{(r)}N_{i,k+1-r}(x)
    \end{equation*}
    met $c_i^{(0)} = c_i$ en $c_i^{(r)} = (k+1-r)\frac{c_i^{(r-1)} - c_{i-1}^{(r-1)}}{t_{i+k+1-r}-t_i}$. Indien $r=k$, dan geeft bovenstaande formule de rechterafgeleide.
\end{lem}

\begin{prf}[Differentiëren van een splinefunctie]{prf:diff_spline}
    We bewijzen per inductie, beginnende bij het basisgeval $r=1$. 
    \begin{align*}
        s'(x) 
            &= \sum_{i=-k}^{n-1}c_i^{(0)}N'_{i,k+1}(x) \\
            &= \sum_{i=-k}^{n-1}c_i^{(0)}k
                \left(
                    \frac{N_{i,k}(x)}{t_{i+k}-t_i} - \frac{N_{i+1,k}(x)}{t_{i+k+1}-t_{i+1}}
                \right) \\
            &= \sum_{i=-(k-1)}^{n-1}c_i^{(1)}N_{i,k}(x) \quad \text{met} \quad c_i^{(1)} = k \frac{c_i^{(0)} - c_{i-1}^{(0)}}{t_{i+k} - t_i}
    \end{align*}
    \textcolor{brown}{
        Stel nu dat het geldt voor $r - 1$, namelijk:
        \begin{equation*}
            s^{(r-1)}(x) = \sum_{i=-(k+1-(r-1))}^{n-1}c_i^{(r-1)}N_{i,k+1-(r-1)}(x)
        \end{equation*} 
        Door deze expressie af te leiden, kunnen we aantonen dat het ook geldt voor $r$ wegens inductie:
        \begin{align*}
            s^{(r)}(x)
                &= \sum_{i=-(k+1-(r-1))}^{n-1}c_i^{(r-1)}N'_{i,k+1-(r-1)}(x) \\
                &= \sum_{i=-(k+1-r)}^{n-1}c_i^{(r-1)}(k+1-r)
                \left(
                    \frac{N_{i,k+1-r}(x)}{t_{i+k+1-r}-t_i} - \frac{N_{i+1,k+1-r}(x)}{t_{i+k+1-(r-1)}-t_{i+1}}
                \right) \\
                &= \sum_{i=-(k+1-r)}^{n-1}c_i^{(r)}
                N_{i,k+1-r}(x) \quad \text{met} \quad c_i^{(r)} =(k+1-r) \frac{c_i^{(r-1)} - c_{i-1}^{(r-1)}}{t_{i+k+1-r} - t_i}
        \end{align*}
    }
\end{prf}